\documentclass[journal,12pt,twocolumn]{IEEEtran}
\usepackage{setspace}
\usepackage{gensymb}
\singlespacing
\usepackage[cmex10]{amsmath}
\usepackage{amsthm}
\usepackage{mathrsfs}
\usepackage{txfonts}
\usepackage{stfloats}
\usepackage{bm}
\usepackage{cite}
\usepackage{cases}
\usepackage{subfig}
\usepackage{longtable}
\usepackage{multirow}
\usepackage{enumitem}
\usepackage{mathtools}
\usepackage{tikz}
\usepackage{circuitikz}
\usepackage{verbatim}
\usepackage[breaklinks=true]{hyperref}
\usepackage{tkz-euclide} % loads  TikZ and tkz-base
\usepackage{listings}
\usepackage{color}    
\usepackage{array}    
\usepackage{longtable}
\usepackage{calc}     
\usepackage{multirow} 
\usepackage{hhline}   
\usepackage{ifthen}   
\usepackage{lscape}
\usepackage{chngcntr}
\DeclareMathOperator*{\Res}{Res}
\renewcommand\thesection{\arabic{section}}
\renewcommand\thesubsection{\thesection.\arabic{subsection}}
\renewcommand\thesubsubsection{\thesubsection.\arabic{subsubsection}}

\renewcommand\thesectiondis{\arabic{section}}
\renewcommand\thesubsectiondis{\thesectiondis.\arabic{subsection}}
\renewcommand\thesubsubsectiondis{\thesubsectiondis.\arabic{subsubsection}}
\renewcommand\thetable{\arabic{table}}
% correct bad hyphenation here
\hyphenation{op-tical net-works semi-conduc-tor}
\def\inputGnumericTable{}                                 %%

\lstset{
language=bash,
frame=single, 
breaklines=true,
columns=fullflexible,
basicstyle=\ttfamily,
}
%\lstset{
%language=tex,
%frame=single, 
%breaklines=true
%}

\DeclareMathOperator*{\argmax}{arg\,max}
\DeclareMathOperator*{\argmin}{arg\,min}
\begin{document}
\newtheorem{theorem}{Theorem}[section]
\newtheorem{problem}{Problem}
\newtheorem{proposition}{Proposition}[section]
\newtheorem{lemma}{Lemma}[section]
\newtheorem{corollary}[theorem]{Corollary}
\newtheorem{example}{Example}[section]
\newtheorem{definition}[problem]{Definition}
\newcommand{\BEQA}{\begin{eqnarray}}
\newcommand{\EEQA}{\end{eqnarray}}
\newcommand{\define}{\stackrel{\triangle}{=}}
\bibliographystyle{IEEEtran}
\providecommand{\mbf}{\mathbf}
\providecommand{\pr}[1]{\ensuremath{\Pr\left(#1\right)}}
\providecommand{\qfunc}[1]{\ensuremath{Q\left(#1\right)}}
\providecommand{\sbrak}[1]{\ensuremath{{}\left[#1\right]}}
\providecommand{\lsbrak}[1]{\ensuremath{{}\left[#1\right.}}
\providecommand{\rsbrak}[1]{\ensuremath{{}\left.#1\right]}}
\providecommand{\brak}[1]{\ensuremath{\left(#1\right)}}
\providecommand{\lbrak}[1]{\ensuremath{\left(#1\right.}}
\providecommand{\rbrak}[1]{\ensuremath{\left.#1\right)}}
\providecommand{\cbrak}[1]{\ensuremath{\left\{#1\right\}}}
\providecommand{\lcbrak}[1]{\ensuremath{\left\{#1\right.}}
\providecommand{\rcbrak}[1]{\ensuremath{\left.#1\right\}}}
\theoremstyle{remark}
\newtheorem{rem}{Remark}
\newcommand{\sgn}{\mathop{\mathrm{sgn}}}
\providecommand{\abs}[1]{\left\vert#1\right\vert}
\providecommand{\res}[1]{\Res\displaylimits_{#1}} 
\providecommand{\norm}[1]{\left\lVert#1\right\rVert}
\providecommand{\mtx}[1]{\mathbf{#1}}
\providecommand{\mean}[1]{E\left[ #1 \right]}   
\providecommand{\fourier}{\overset{\mathcal{F}}{ \rightleftharpoons}}
\providecommand{\system}[1]{\overset{\mathcal{#1}}{ \longleftrightarrow}}
\newcommand{\solution}{\noindent \textbf{Solution: }}
\newcommand{\cosec}{\,\text{cosec}\,}
\providecommand{\dec}[2]{\ensuremath{\overset{#1}{\underset{#2}{\gtrless}}}}
\newcommand{\myvec}[1]{\ensuremath{\begin{pmatrix}#1\end{pmatrix}}}
\newcommand{\mydet}[1]{\ensuremath{\begin{vmatrix}#1\end{vmatrix}}}
\renewcommand{\vec}[1]{\boldsymbol{\mathbf{#1}}}
\def\putbox#1#2#3{\makebox[0in][l]{\makebox[#1][l]{}\raisebox{\baselineskip}[0in][0in]{\raisebox{#2}[0in][0in]{#3}}}}
     \def\rightbox#1{\makebox[0in][r]{#1}}
     \def\centbox#1{\makebox[0in]{#1}}
     \def\topbox#1{\raisebox{-\baselineskip}[0in][0in]{#1}}
     \def\midbox#1{\raisebox{-0.5\baselineskip}[0in][0in]{#1}}

\vspace{3cm}
\title{Visualizaing the Internet Topology}
\author{Anshul Sangrame (CS21BTECH11004)\\Gautam Singh (CS21BTECH11018)\\Varun Gupta (CS21BTECH11060)}
\maketitle
\tableofcontents
\bigskip

This document contains a report on the Python library \texttt{netvis} created by
the authors, to visualize the topology of the internet. It covers some
implementation details, challenges in implementation and conclusions from using
this library.

\section{Implementation}

\texttt{netvis} is a \texttt{python} package, installable using \texttt{pip}. In
the following subsections, we describe the installation and sailent features
provided by this library.

The entire source code is hosted on
\href{https://github.com/goats-9/cs3530-assignments/tree/main/netvis}{GitHub}.
The verbose implementation details are provided as comments in the source code,
which have been omitted here for brevity.

\subsection{Installation}

\texttt{netvis} can be installed by typing the following in a terminal window.

\begin{lstlisting}
$ git clone https://github.com/goats-9/cs3530-assignments
$ cd cs3530-assignments/netvis
$ pip install -e .
\end{lstlisting}

It is \emph{highly recommended} to install the package in a Python virtual
environment.

\subsection{Requirements}

To help in the visualization of the topology of the Internet, \texttt{netvis}
depends on the following Python packages.
\begin{enumerate}
     \item \href{https://pandas.pydata.org/}{\texttt{pandas}}. To process the
     raw data obtained for visualization.
     \item \href{https://openpyxl.readthedocs.io/en/stable/}{\texttt{openpyxl}}.
     To handle raw data in Excel files.
     \item \href{https://pyvis.readthedocs.io/en/latest/}{\texttt{pyvis}}. To
     visualize the network in the form of a graph.
\end{enumerate}

Additionally, note that this library is suitable for Linux systems only, and
also requires the executable
\href{https://linux.die.net/man/8/mtr}{\texttt{mtr}} to be installed on the
system.

\subsection{Features}

The sailent features of \texttt{netvis} are the following.
\begin{enumerate}
     \item A \texttt{NetGraph} object to abstract the topology of the Internet
     into a directed graph, containing a list of nodes and edges.
     \item Use of \texttt{mtr} (short for \emph{my traceroute}) with appropriate
     arguments to obtain the entire sequence of hops from source to destination,
     along with AS numbers of ISPs (if available) in CSV form for easy data
     handling.
     \item Provisions to save and load \texttt{NetGraph} objects to and from
     Excel files so that they can be used in other Python scripts.
     \item Provision to create a union of two \texttt{NetGraph} objects so that
     the topology from multiple sources to multiple destinations is
     \emph{integrated}.
     \item Rendering of the graph depicting the topology in HTML, along with the
     use of CSS and JS to make the graph interactive as well as show the legend.
     The graph is present in \texttt{graph.html}, but the integrated legend is
     displayed on rendering \texttt{index.html}.
\end{enumerate}

\section{Challenges}

The authors wanted to automate the entire process of visualizing the internet
topology. The following challenges were faced in the process.

\begin{enumerate}
     \item Whenever \texttt{netvis} is imported, a dataset about 30 MB in size
     is downloaded and loaded into a \texttt{DataFrame} object. This may not be
     suitable for systems with constraints on resources like memory or internet
     speed (such as embedded systems).

     A more suitable solution could have been hosting the data on a webserver
     and making HTTP GET requests to a URL.
     \item \texttt{pyvis} does not have a provision to provide a legend for easy
     use of the generated graph. The authors worked around this using JavaScript
     to create a legend.
     \item The outputs of \texttt{mtr} did not include the source IP address. It
     was manually added by making use of the Python standard library module
     \href{https://docs.python.org/3/library/socket.html}{\texttt{socket}}.
\end{enumerate}

\section{Observations}
The following conclusions can be made based on the data collected.

\begin{enumerate}
     \item From the raw data, sites hosted in countries east of India have
     lesser RTTs (less than 200 ms) compared to sites hosted in countries to the
     west of India (at least 200 ms).
     \item The maximum RTT on average was experienced while pinging the site
     \href{https://www.hi.is/}{\texttt{hi.is}}, which makes sense since the site
     was hosted in Iceland. On the other hand, the minimum RTT was experienced
     while pinging \href{https://www.isro.gov.in/}{\texttt{www.isro.gov.in}},
     which is hosted in India, where all the source devices are.
     \item 
     \begin{enumerate}
          \item Using the LAN service available in IITH (source IP
          \texttt{10.5.80.104}), the packets always travelled through the ISP at
          IITH and sometimes through the NKN Core Network.
          \item Using mobile data, the packets always went through the service
          providers (Airtel and Reliance Jio).
          \item Using personal WiFi network (not in campus, source IP
          \texttt{192.168.1.38}), the packets went through
          \href{https://www.ctrls.in/}{CtrlS datacenters}, which might mean that
          the \href{https://reachtele.net/}{local ISP} uses the services of this
          datacenter to host and provide Internet services to their customers.
     \end{enumerate}
     \item By looking up the AS numbers, most of the sites chosen as
     destinations are hosted in the USA (even those with non-American domains).
\end{enumerate}

\section{Resources}

\begin{enumerate}
     \item The raw and processed data is present in the \texttt{data} directory.
     \item The scripts used to collect and visualize the data are present in the
     \texttt{tests} directory.
     \item The ASN dataset \texttt{data/asn.tsv} was taken and edited from
     \href{https://asntool.com/}{\texttt{asntool.com}}.
     \item The \href{https://hatch.pypa.io/latest/}{hatch} build system was used
     to package the code.
\end{enumerate}

\end{document}
